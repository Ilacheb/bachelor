\addcontentsline{toc}{section}{Литература}

\label{section_bibliography}

\begin{thebibliography}{99} % Для того, чтобы цифры были одинаково выровнены, говорим, что максимальное число будет двухзначное

\bibitem{besekerskiy_1}
Бесекрерский В.А., Попов Е.П. Теория автоматического управления 1. СПб.: Профессия, 2003.

\bibitem{besekerskiy_2}
Бесекрерский В.А., Попов Е.П. Теория автоматического управления 2. СПб.: Профессия, 2003.

\bibitem{besekerskiy_3}
Бесекрерский В.А., Попов Е.П. Теория автоматического управления 3. СПб.: Профессия, 2003.

\bibitem{besekerskiy_4}
Бесекрерский В.А., Попов Е.П. Теория автоматического управления 4. СПб.: Профессия, 2003.

\bibitem{besekerskiy_5}
Бесекрерский В.А., Попов Е.П. Теория автоматического управления 5. СПб.: Профессия, 2003.

\bibitem{besekerskiy_6}
Бесекрерский В.А., Попов Е.П. Теория автоматического управления 6. СПб.: Профессия, 2003.

\bibitem{besekerskiy_7}
Бесекрерский В.А., Попов Е.П. Теория автоматического управления 7. СПб.: Профессия, 2003.

\bibitem{besekerskiy_8}
Бесекрерский В.А., Попов Е.П. Теория автоматического управления 8. СПб.: Профессия, 2003.

\bibitem{besekerskiy_9}
Бесекрерский В.А., Попов Е.П. Теория автоматического управления 9. СПб.: Профессия, 2003.

\bibitem{besekerskiy_10}
Бесекрерский В.А., Попов Е.П. Теория автоматического управления 10. СПб.: Профессия, 2003.
\end{thebibliography}