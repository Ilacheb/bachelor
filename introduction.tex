\section*{Введение} % Команда со звездочкой добавляет не нумерованный раздел
\addcontentsline{toc}{section}{Введение} % Добавляем ненумерованный раздел в содержание
\label{section_introduction} %содаем метку, чтобы далее ссылаться на данный раздел

Далее три параграфа несвязного текста. Разделяются параграфы в исходниках просто пустой строкой.

\uppercase{И тут сразу теорема как выскочит, посреди введения!}

\begin{theorem}
	Какая-то теорема.
\end{theorem}

\begin{proof}
	"Где пруфы, Билли?"
\end{proof}

Волна упруго облучает термодинамический газ без обмена зарядами или спинами. Интерпретация всех изложенных ниже наблюдений предполагает, что еще до начала измерений среда индуцирует гравитационный кристалл по мере распространения сигнала в среде с инверсной населенностью. Вещество синхронизует гравитационный взрыв как при нагреве, так и при охлаждении. Фонон возбудим. Взвесь усиливает взрыв, при этом дефект массы не образуется.

Если для простоты пренебречь потерями на теплопроводность, то видно, что течение среды последовательно. Течение среды упруго поглощает спиральный погранслой, и это неудивительно, если вспомнить квантовый характер явления. Газ стабилизирует гравитационный гидродинамический удар в том случае, когда процессы переизлучения спонтанны. Волновая тень отражает субсветовой лазер, даже если пока мы не можем наблюсти это непосредственно.

Разрыв переворачивает циркулирующий атом по мере распространения сигнала в среде с инверсной населенностью. Тело, в отличие от классического случая, излучает фотон при любом их взаимном расположении. Квантовое состояние отражает поток вне зависимости от предсказаний самосогласованной теоретической модели явления. Ударная волна трансформирует квант в том случае, когда процессы переизлучения спонтанны. Мишень волнообразна.

% если надо вручную указать, что последующий текст должен быть с новой страницы, пишем:
\newpage

В данном месте в коде показано, как вручную указать, что следующий параграф стоит начать с новой страницы. Разделы автоматически начинаются с новой страницы, в соответствии с ГОСТом.

Так вставляются формулы в текст $\omega$.

Пример ссылки на литературу \cite{besekerskiy_1, besekerskiy_2, besekerskiy_3, besekerskiy_9}.