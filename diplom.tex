\documentclass[a4paper]{article}

\usepackage{polyglossia}  % многоязычная вёрстка

\usepackage{diplom}		  % штампы, настройки шрифтов и прочее для диплома
\usepackage{makecell}	  % для работы с выравниваем в таблицах
\usepackage{multirow}     % разбивка ячейки на несколько рядов
\usepackage{ulem}		  % различные виды подчёркиваний
\usepackage{indentfirst}  % постоянно делать отступ красной строки
						  % для нового параграфа
\usepackage{hyperref}	  % гипертекст и перекрёстные ссылки

% Расширенные наборы математических символов
\usepackage{amssymb,amsmath,amsfonts,latexsym,mathtext}

% Устанавливает главный язык документа
\setdefaultlanguage[spelling=modern]{russian}
% Объявляет второй язык документа
\setotherlanguage{english}

% Свойства шрифтов по умолчанию
\defaultfontfeatures{Ligatures={TeX}}

\setmainfont[Ligatures=TeX]{Times New Roman}  % основной шрифт документа
\setsansfont{PT Sans}  % шрифт без засечек
\setmonofont{Consolas}  % моноширинный шрифт


\makeatletter

	% Глубина разделов, попадающих в содержание
	\setcounter{tocdepth}{2}

	\renewcommand\theadfont{\normalsize}
	
	%% Допущение
\newtheorem{assumption}{\hspace{\parindent}Д\,о\,п\,у\,щ\,е\,н\,и\,е\,}

%% Допущение
\newtheorem{purpose}{\hspace{\parindent}Ц\,е\,л\,ь\, \ \,у\,п\,р\,а\,в\,л\,е\,н\,и\,я\,}

%%  Теорема
\newtheorem{theorem}{\hspace{\parindent}\sl{Т\,е\,о\,р\,е\,м\,а\,}}

%%  Следствие 
\newtheorem{corollary}{\hspace{\parindent}\sl{С\,л\,е\,д\,с\,т\,в\,и\,е\,}}

%%  Лемма
\newtheorem{lemma}{\hspace{\parindent}\sl{Л\,е\,м\,м\,а\,}}

%%  Утверждение
\newtheorem{statement}{\hspace{\parindent}\sl{У\,т\,в\,е\,р\,ж\,д\,е\,н\,и\,е\,}}

%%  Предложение
\newtheorem{proposition}{\hspace{\parindent}\sl{П\,р\,е\,д\,л\,о\,ж\,е\,н\,и\,е\,}}

%%  Определение
\newtheorem{definition}{\hspace{\parindent}\sl{О\,п\,р\,е\,д\,е\,л\,е\,н\,и\,е\,}}

%%  Задача
\newtheorem{problem}{\hspace{\parindent}\sl{З\,а\,д\,а\,ч\,а\,}}

%%  Доказательство
\newenvironment{proof}
{\vspace{1pt}\par{\sl% 
Д\,о\,к\,а\,з\,а\,т\,е\,л\,ь\,с\,т\,в\,о.\,\ }}%
{\noindent\vspace{1pt}} 

%%  Доказательство теоремы
\newenvironment{proofoftheorem}[1]
{\vspace{1pt}\par{\sl% 
Д\,о\,к\,а\,з\,а\,т\,е\,л\,ь\,с\,т\,в\,о\ \ т\,е\,о\,р\,е\,м\,ы\, #1.}}%
{\noindent\vspace{1pt}} 

%%  Доказательство следствия
\newenvironment{proofofcorollary}[1]
{\vspace{1pt}\par{\sl% 
Д\,о\,к\,а\,з\,а\,т\,е\,л\,ь\,с\,т\,в\,о\  с\,л\,е\,д\,с\,т\,в\,и\,я\, #1.}}%
{\noindent\vspace{1pt}} 

%%  Доказательство леммы
\newenvironment{proofoflemma}[1]
{\vspace{1pt}\par{\sl% 
Д\,о\,к\,а\,з\,а\,т\,е\,л\,ь\,с\,т\,в\,о\ \  л\,е\,м\,м\,ы\, #1.}}%
{\noindent\vspace{1pt}} 

%%  Доказательство утверждения
\newenvironment{proofofstatement}[1]
{\vspace{1pt}\par{\sl% 
Д\,о\,к\,а\,з\,а\,т\,е\,л\,ь\,с\,т\,в\,о\ \ у\,т\,в\,е\,р\,ж\,д\,е\,н\,и\,я\, #1.}}%
{\noindent\vspace{1pt}} 

%%  Доказательство предложения
\newenvironment{proofofproposition}[1]
{\vspace{1pt}\par{\sl% 
Д\,о\,к\,а\,з\,а\,т\,е\,л\,ь\,с\,т\,в\,о\, п\,р\,е\,д\,л\,о\,ж\,е\,н\,и\,я\, #1.}}%
{\noindent\vspace{1pt}} 

%%  Алгоритм
\newtheorem{alg}{\hspace{\parindent}\sl{А\,л\,г\,о\,р\,и\,т\,м\,}}%
\newenvironment{algorithm}[1][\unskip]{\begin{alg}[#1]\em}{\end{alg}}%
\def\thealgorithm{\thealg}

%%  Замечание
\newtheorem{rem}{\hspace{\parindent}\sl{З\,а\,м\,е\,ч\,а\,н\,и\,е}}
\newenvironment{remark}[1][\unskip]{\begin{rem}[#1]\em}{\end{rem}}%
\def\theremark{\therem}

%%  Пример
\newtheorem{exmpl}{\hspace{\parindent}\sl{П\,р\,и\,м\,е\,р\,}}%
\newenvironment{example}[1][\unskip]{\begin{exmpl}[#1]\em}{\end{exmpl}}%
\def\theexample{\theexmpl}


 % настройка некоторых окружений
	\linespread{1.3}    % настройка межстрочного интервала
	\tolerance=1000     % настройка чувствительности вставки переносов
	\hfuzz=0pt
	\sloppy
	
	\graphicspath{{images/}}  % путь к папке с картинками
	
	% Данные для заполнения рамок
	\titleDepartment{КТиУ}
	\titleSchollOf{220200}
	\titleDegree{бакалавр техники и технологии}
	\titleСhief{Специализация}
	\titleSubDepartment{КСУИ}
	\gostklgi{КСУИ.0144147.001 ПЗ}
	\gostrazrabotchik{Фамилия И.О}  % студента
	\gostproveril{Фамилия И.О.}  % проверяющего
	\gostnormokontroler{}
	\gostutverdil{}
	\gosttitledocument{Большое и труднопроизносимое, для пущей важности, название работы}
	\gosttitlecompany{СПбГУ ИТМО\\Кафедра СУИ}
	\gostlitera{д}  % буква в таблице, ставится только в дипломной работе, в УИРС не надо
	\date{}

\makeatother


\begin{document}
	% Переменование "Список литературы" в "Литература"
	\renewcommand{\refname}{Литература}
	
	% Первая страница это Титульный лист
	% \maketitle команда отключена, так как титульный лист дается на кафедре
	% Вторая страница это задание
	% Задание
	% Поэтому сам диплом начинается с третьей страницы
	\setcounter{page}{3}
	% Содержание
	\tableofcontents
	% Примеры разделов ПЗ
	\section*{Введение} % Команда со звездочкой добавляет не нумерованный раздел
\addcontentsline{toc}{section}{Введение} % Добавляем ненумерованный раздел в содержание
\label{section_introduction} %содаем метку, чтобы далее ссылаться на данный раздел

Далее три параграфа несвязного текста. Разделяются параграфы в исходниках просто пустой строкой.

\uppercase{И тут сразу теорема как выскочит, посреди введения!}

\begin{theorem}
	Какая-то теорема.
\end{theorem}

\begin{proof}
	"Где пруфы, Билли?"
\end{proof}

Волна упруго облучает термодинамический газ без обмена зарядами или спинами. Интерпретация всех изложенных ниже наблюдений предполагает, что еще до начала измерений среда индуцирует гравитационный кристалл по мере распространения сигнала в среде с инверсной населенностью. Вещество синхронизует гравитационный взрыв как при нагреве, так и при охлаждении. Фонон возбудим. Взвесь усиливает взрыв, при этом дефект массы не образуется.

Если для простоты пренебречь потерями на теплопроводность, то видно, что течение среды последовательно. Течение среды упруго поглощает спиральный погранслой, и это неудивительно, если вспомнить квантовый характер явления. Газ стабилизирует гравитационный гидродинамический удар в том случае, когда процессы переизлучения спонтанны. Волновая тень отражает субсветовой лазер, даже если пока мы не можем наблюсти это непосредственно.

Разрыв переворачивает циркулирующий атом по мере распространения сигнала в среде с инверсной населенностью. Тело, в отличие от классического случая, излучает фотон при любом их взаимном расположении. Квантовое состояние отражает поток вне зависимости от предсказаний самосогласованной теоретической модели явления. Ударная волна трансформирует квант в том случае, когда процессы переизлучения спонтанны. Мишень волнообразна.

% если надо вручную указать, что последующий текст должен быть с новой страницы, пишем:
\newpage

В данном месте в коде показано, как вручную указать, что следующий параграф стоит начать с новой страницы. Разделы автоматически начинаются с новой страницы, в соответствии с ГОСТом.

Так вставляются формулы в текст $\omega$.

Пример ссылки на литературу \cite{besekerskiy_1, besekerskiy_2, besekerskiy_3, besekerskiy_9}. % подцепляет файл intoduction.tex
	\section{Обзор информационных источников по рассматриваемой проблеме} % Добавление нумерованного раздела, он сам попадает в содержание
\label{section_overview}

\subsection{Обзор методов управления в условиях запаздывания} % Добавляем подраздел
\label{overview_delays}

Так делаются ссылки на рисунки ~\ref{fig:pid_1}. Ниже показан пример вставки двух изображений, правда в подписи буквы "a" и "б" прописаны вручную.

\begin{figure}[H]
	\begin{minipage}[t]{0.56\textwidth}
		\center{
			\textbf{a}
			\includegraphics[width=\textwidth,height=130pt]{overview/sim_pid_1}
		}
	\end{minipage}
	\hfill
	\begin{minipage}[t]{0.43\textwidth}
		\center{
			\textbf{б}\vspace{10pt}
			\includegraphics[width=\textwidth,height=130pt]{overview/pid_1}
		}
	\end{minipage}
	\vspace{-7mm}
	\caption{Общее название иллюстраций: a --- название первого рисунка; б~---~название второго рисунка}
	\label{fig:pid_1}
\end{figure}

В исходниках часто можно увидеть символ $\sim$ это не отображаемый символ, обозначающий, что переносить в этом месте нельзя( пример, подпись рисунка \ref{fig:pid_1}.б). Так как он не отображаемый, тут выведен математический знак на него похожий.

\subsubsection{Еще один подраздел}
\label{subsubsection_smith}

Мишень масштабирует наносекундный пульсар в том случае, когда процессы переизлучения спонтанны. Электрон, на первый взгляд, теоретически возможен. Фронт индуцирует солитон вне зависимости от предсказаний самосогласованной теоретической модели явления. Возмущение плотности, на первый взгляд, стабилизирует квазар, однозначно свидетельствуя о неустойчивости процесса в целом. Электрон квантово разрешен. Лептон противоречиво ускоряет лазер, как и предсказывает общая теория поля.

Газ усиливает электронный гидродинамический удар, хотя этот факт нуждается в дальнейшей тщательной экспериментальной проверке. Кристалл инвариантен относительно сдвига. При наступлении резонанса гравитирующая сфера когерентно концентрирует экранированный погранслой, что лишний раз подтверждает правоту Эйнштейна. Примесь эксперментально верифицируема.

Изолируя область наблюдения от посторонних шумов, мы сразу увидим, что зеркало нейтрализует пульсар при любом агрегатном состоянии среды взаимодействия. Неоднородность скалярна. Взрыв когерентно трансформирует вихревой лептон, но никакие ухищрения экспериментаторов не позволят наблюдать этот эффект в видимом диапазоне. Зеркало противоречиво возбуждает магнит, хотя этот факт нуждается в дальнейшей тщательной экспериментальной проверке. Гравитирующая сфера облучает кристалл, поскольку любое другое поведение нарушало бы изотропность пространства.

Пример вставки одиночного изображения.
\begin{figure}[H]
	\vspace{5mm}
	\center{
		\includegraphics[width=0.8\textwidth]{overview/sim_pid_2}
	}	
	\caption{Название рисунка}
	\label{fig:smith_predictor}
\end{figure}

Пример вставки формулы:
\begin{equation}
	\label{smith_ideal}
	y=P_0e^{-sh}\left(\frac{R}{1+RM_0}\right)r
		=\left(\frac{P_0R}{1+RP_0}e^{-sh}\right)r.
\end{equation}

	% \include{definition} и т.д создаете столько файлов и разделов, сколько нужно
	% \include{tests}
	\section*{Заключение}
\addcontentsline{toc}{subsection}{Заключение}
\label{section_conclusion}

Течение среды, вследствие квантового характера явления, притягивает квантово-механический фотон независимо от расстояния до горизонта событий. Вселенная эксперментально верифицируема. Газ, в согласии с традиционными представлениями, ускоряет барионный резонатор, и это неудивительно, если вспомнить квантовый характер явления. Многочисленные расчеты предсказывают, а эксперименты подтверждают, что расслоение искажает взрыв даже в случае сильных локальных возмущений среды. Изолируя область наблюдения от посторонних шумов, мы сразу увидим, что объект квантуем. Мишень, вследствие квантового характера явления, инструментально обнаружима.

Зеркало, как бы это ни казалось парадоксальным, когерентно поглощает адронный квазар, и это неудивительно, если вспомнить квантовый характер явления. Фонон, если рассматривать процессы в рамках специальной теории относительности, отражает ультрафиолетовый магнит - все дальнейшее далеко выходит за рамки текущего исследования и не будет здесь рассматриваться. Солитон, как можно показать с помощью не совсем тривиальных вычислений, расщепляет гамма-квант, даже если пока мы не можем наблюсти это непосредственно. Тело представляет собой ультрафиолетовый погранслой - все дальнейшее далеко выходит за рамки текущего исследования и не будет здесь рассматриваться. Излучение притягивает магнит, однозначно свидетельствуя о неустойчивости процесса в целом.

При наступлении резонанса среда индуцирует изобарический гидродинамический удар, и этот процесс может повторяться многократно. Течение среды, при адиабатическом изменении параметров, ускоряет межатомный сверхпроводник даже в случае сильных локальных возмущений среды. Экситон трансформирует атом в том случае, когда процессы переизлучения спонтанны. Жидкость выталкивает экзотермический экситон в том случае, когда процессы переизлучения спонтанны.
	\addcontentsline{toc}{section}{Литература}

\label{section_bibliography}

\begin{thebibliography}{99} % Для того, чтобы цифры были одинаково выровнены, говорим, что максимальное число будет двухзначное

\bibitem{besekerskiy_1}
Бесекрерский В.А., Попов Е.П. Теория автоматического управления 1. СПб.: Профессия, 2003.

\bibitem{besekerskiy_2}
Бесекрерский В.А., Попов Е.П. Теория автоматического управления 2. СПб.: Профессия, 2003.

\bibitem{besekerskiy_3}
Бесекрерский В.А., Попов Е.П. Теория автоматического управления 3. СПб.: Профессия, 2003.

\bibitem{besekerskiy_4}
Бесекрерский В.А., Попов Е.П. Теория автоматического управления 4. СПб.: Профессия, 2003.

\bibitem{besekerskiy_5}
Бесекрерский В.А., Попов Е.П. Теория автоматического управления 5. СПб.: Профессия, 2003.

\bibitem{besekerskiy_6}
Бесекрерский В.А., Попов Е.П. Теория автоматического управления 6. СПб.: Профессия, 2003.

\bibitem{besekerskiy_7}
Бесекрерский В.А., Попов Е.П. Теория автоматического управления 7. СПб.: Профессия, 2003.

\bibitem{besekerskiy_8}
Бесекрерский В.А., Попов Е.П. Теория автоматического управления 8. СПб.: Профессия, 2003.

\bibitem{besekerskiy_9}
Бесекрерский В.А., Попов Е.П. Теория автоматического управления 9. СПб.: Профессия, 2003.

\bibitem{besekerskiy_10}
Бесекрерский В.А., Попов Е.П. Теория автоматического управления 10. СПб.: Профессия, 2003.
\end{thebibliography}
\end{document}